\documentclass[11pt]{article}

% \usepackage{ngerman} 
\usepackage[german]{babel} 
\usepackage{ucs} 
\usepackage[utf8x]{inputenc} 
\usepackage{vmargin}
\usepackage{amsmath,amssymb,amstext} 
\usepackage{graphicx}
\usepackage{fancyhdr}
\usepackage{latexsym}
\usepackage{datetime}

\pagestyle{fancy}
\fancyhf{} % Clear all header and footer fields
\fancyhead[L]{\nouppercase{\leftmark}} % Section title on the left, not subsection
\fancyhead[R]{\thepage} % Page number on the right

\parindent0pt
\parskip1ex
\setlength{\headheight}{15pt}

\title{Linux-Kurzskriptum} 
\author{Edgar Neukirchner} 
\newdateformat{mydate}{\the\year.\twodigit\month.\twodigit\day}
\date{\mydate\today\ Bulme Graz} 

% new command \keys{xxx} 

\newbox\mybox
\newcount\laenge
\newcount\laengehalbe
\newcommand{\keys}[1]{%
\setbox\mybox=\hbox{\footnotesize #1}%
\laenge=\wd\mybox%
\advance\laenge by 400000% entspricht 2.15 mm
\laengehalbe=\laenge%
\divide \laengehalbe by 2%
\unitlength1sp\begin{picture}(\laenge,600000)(0, 140000)
\put(\laengehalbe, 300000){\oval(\laenge, 600000)}%
\put(200000, 140000){\unhbox\mybox}
\end{picture}}

\begin{document}

\maketitle
\newpage
\tableofcontents
\newpage

\section{Einstieg} 
\subsection{Was ist ein Betriebssystem?}
Ein Computersystem lässt sich schematisch in einer Art Schalenstruktur
darstellen: Die Rechnerhardware findet sich im Zentrum, umgeben vom
Kernel mit seinen Modulen. Darauf folgen Standardbibliotheken und
schließlich Anwenderprogramme.

\subsection{Was ist Linux?} 
Linux ist ein freies Unix-ähnliches Betriebssystem. Das ``eigentliche''
Linux ist nur der Betriebssystemkernel (entwickelt von Linus Torvalds).
Um mit dem System tatsächlich arbeiten zu können, sind eine große Anzahl
von System- und Hilfsprogrammen notwendig; die meisten von ihnen sind
freie Software (GPL).

Die Zusammenstellung Kernel plus Software heißt Linux-Distribution. Distributionen unterscheiden sich im Wesentlichen durch das Veröffentlichungs-Modell (Release model fixed oder rolling) und durch das zur Verwaltung der Software verwendete Programm. Debian, Ubuntu und Mint verwenden den Debian Package Manager, Redhat, Fedora, Suse und Almalinux den von Redhat. Rolling releases wie Arch oder Manjaro sind eher für Entwickler gedacht, die stets brandaktuelle Software benötigen.
Hinter vielen Distributionen stehen Firmen (Redhat, Suse), die wahlweise freie oder kostenpflichtige Versionen (mit Support und Zertifizierung) anbieten. Reine Community-Distributionen sind beispielsweise Debian Arch Linux.

Linux ist nicht das einzige freie Betriebssystem: FreeBSD, OpenBSD und
NetBSD sind ebenfalls freie Unix-Derivate, wobei die Entwicklung in
straffer geführten Projektteams stattfindet. Mac OS X ist zumindest
teilweise ebenfalls BSD-basiert (Darwin).

\subsection{Charakteristika von Unix/Linux}

\begin{itemize}
\item hierarchisches Dateisystem
\item multitaskingfähig (mehrere Aufgaben gleichzeitig)
\item multiuserfähig (mehrere Benutzer gleichzeitig) 
\item netzwerkfähig
\item modular aufgebaut: Benutzer bzw. Administrator können die
Komponenten frei kombinieren. Graphische Oberfläche (Windowmanager mit Desktopumgebung)
und Betriebssystem sind völlig voneinander getrennt - ein Programm kann
so auf einem Rechner gestartet und auf einem anderen angezeigt werden.
\item offen (Quellcode im Internet verfügbar)
\end{itemize}

Für den Anwender unmittelbar wichtige Unterschiede zu Windows/DOS:
\begin{itemize}
\item login mit Passworteingabe erforderlich
\item bestimmte Dateien und Verzeichnisse sind  nicht für alle Benutzer
les- und schreibbar.
\item Groß- und Kleinschreibung wichtig: \texttt{Meinfile} und
\texttt{meinfile} sind unterschiedliche Dateien!
\item Pfad-Trenner ist bei DOS der Rückwärts-Schrägstrich $ \backslash
$, bei Unix der Vorwärts-Schrägstrich /. Unter Unix werden mit $
\backslash $ lange Befehlszeilen in der nächsten Zeile fortgesetzt.
\item Es gibt keine Laufwerksbuchstaben.
\end{itemize}

\subsection{Linux-Installationsvarianten}
Die ``klassische'' Installation einer Linux-Distribution ist üblicherweise
mit einer Neupartitionierung der Festplatte verbunden. Damit verbunden
ist das Risiko des versehentlichen Löschens vorhandener Daten - daher
Backup nicht vergessen! Eine Installation als virtualisierter Gast eines vorhandenen Systems vermeidet diese Risiken,allerdings  um den Preis von Performance-Verlusten.

Beispiele für Virtualisierungs-Programme: 
\begin{itemize}
\item virtualbox (Gratisversion erhältlich), 
\item KVM (nur für Linux-Wirtssysteme)
\item Windows Subsystem for Linux (Windows 11 unterstützt bereits die Grafikausgabe von Linux-Gästen).
\end{itemize}

\section{Die Shell}
\subsection{Merkmale der Shell}
\begin{itemize}
\item Hauptaufgabe: Weiterleitung der eingegebenen Kommandozeile ans
Betriebssystem (Kommandozeileninterpreter, vergleiche cmd oder powershell unter Windows)
\item Kombination von Kommandos, z.B. mit if- oder while-Bedingungen.
Auf diese Weise können Kommandoeingaben in Form von Shellscripts
automatisiert werden.
\item Konfigurierbarkeit, daher Möglichkeit der Erstellung einer
persönlichen Arbeitsumgebung
\item Unix ermöglicht dem Benutzer die Auswahl seiner gewünschten Shell,
die gängigsten sind bash (Standard-Shell unter Linux) und zsh.  Um herauszufinden, welche Shell gerade verwendet wird, tippt
man in der Konsole \texttt{echo \$SHELL} ein.
\end{itemize}

\subsection{Kontrollsequenzen und Wildcards der bash}
\begin{tabular}{|l|l|}
\hline
Automatische Kommando-/Dateinamenergänzung: & \keys{TAB} \\
\hline
In der Kommando-History blättern:&\keys{$ \uparrow $}\keys{$\downarrow $} \\
\hline
Inkrementelle Suche in der History: & \keys{Strg}+\keys{R}   \\
\hline
Programmabbruch: & \keys{Strg}+\keys{C} \\
\hline
Programm anhalten (suspend): & \keys{Strg}+\keys{Z} \\
\hline
angehaltenes Programm im Hintergrund laufen lassen: & \texttt{bg} \\
angehaltenes Programm im Vordergrund laufen lassen: & \texttt{fg} \\
\hline
\texttt{kommando} als Hintergrundprozess starten: & \texttt{kommando \&} \\
\hline
Wildcards: & \\
ein oder mehrere beliebige Zeichen (auch Leerzeichen): & \texttt{*} \\
genau ein beliebiges Zeichen: & \texttt{?} \\
alle Buchstaben von a-c: & \texttt{[a-c]} \\
alle Files mit Endung doc oder cpp & \texttt{*.\{doc,cpp\}} \\
\hline
Quotes (verhindern Interpretation als Sonderzeichen): & \\
nur nachfolgendes Zeichen: & $ \backslash $  \\
String mit Leer-/Sonderzeichen außer Shellvariablen (z.B: \$x): & '' '' \\
alle Sonderzeichen: & ' ' \\
\hline
\end{tabular}

\section{Unix-Kommandos}
\subsection{Aufbau eines Unix-Kommandos}
Unix-Kommandos lauten oft wie die Abkürzung der englischen Bezeichnung
ihrer Funktion.  Lange Linux-Kommandozeilen wirken oft kryptisch; bei
genauerer Betrachtung zeigt sich aber, dass sie immer nach folgenem
Prinzip aufgebaut sind:

\texttt{befehl [-o] argument} 

oder:\\
\texttt{befehl [--option-lang] argument}

In eckigen Klammern stehene Optionen  oder Argumente können, aber müssen
nicht verwendet werden. Ob dem Befehl ein Argument folgt oder nicht,
hängt vom Einsatzzweck und vom Kommando ab. 

\subsection{Die wichtigsten Unix-Kommandos in Kurzfassung}
\begin{tabular}{|l|l|}
\hline
Dateien anzeigen (list): & \texttt{ls [DATEI]} \\
mit ausführlichen Informationen: & \texttt{ls -l} \\
auch versteckte Dateien (beispielsweise .bashrc) & \texttt{ls -a} \\
\hline
aktuelles Arbeitsverzeichnis ausgeben(print working directory): & \texttt{pwd} \\
\hline 
Verzeichnis wechseln (change directory) & \texttt{cd VERZEICHNIS} \\ 
ins eigene Heimverzeichnis & \texttt{cd} \\
\hline 
Verzeichnis anlegen (make directory) & \texttt{mkdir VERZEICHNIS} \\
\hline
Datei kopieren (copy) & \texttt{cp QUELLE ZIEL} \\  
Datei(en) in Zielverzeichnis kopieren & \texttt{cp QUELLE... Ziel} \\
\hline
Dateien umbenennen (move) & \texttt{mv QUELLE ZIEL} \\  
Datei(en) in Zielverzeichnis verschieben & \texttt{mv QUELLE... ZIEL} \\ 
\hline
Dateien löschen (remove) & \texttt{rm DATEI} \\
Verteichnis samt Inhalt löschen (Vorsicht!) & \texttt{rm -r DATEI} \\
\hline
Leeres Verzeichnis löschen & \texttt{rmdir VERZEICHNIS} \\
\hline  
Datei auf einmal ausgeben & \texttt{cat DATEI} \\
Datei betrachten (beenden mit \texttt{q}) & \texttt{less DATEI}  \\
\hline
\end{tabular} 

\subsection{Umleitung}
Wenn von der Benutzerin nicht anders angegeben, gehen die von einem
Programm verarbeiteten Daten folgende Wege: 
\begin{itemize}
\item
Benutzer $ \rightarrow $ Programm: Standardeingabe
(stdin), d.h. Lesen von der Konsole 
\item
Programm $ \rightarrow $ Benutzer: Standardausgabe
(stdout), d.h. Schreiben auf die Konsole
\item
Fehlerausgabe: Standardfehlerausgabe (stderr)
\end{itemize}
Durch Anfügen von Umleitungszeichen an das eigentliche Kommando können
Lese- und Schreiboperationen auch über Dateien laufen.

\subsubsection{Standardeingabe und -ausgabe}
\texttt{sort < adressbuch} \\
Die Einträge der Datei \texttt{adressbuch} werden dem Programm
\texttt{sort} zum alphabetischen Sortieren übergeben. Das Resultat
erscheint am Bildschirm. 

\texttt{date > erstellungsdatum} \\
Die Ausgabe des Programmes \texttt{date} wird anstatt auf den
Bildschirm in die Datei \texttt{erstellungsdatum} geschrieben. Falls die
Datei schon vorher existiert hat, wird sie überschrieben. Soll die
Ausgabe an eine Datei angehängt werden, sieht die Zeile so aus:\\
\texttt{date >> erstellungsdatum} 

\texttt{sort < adressbuch > sortiert} \\
Kombination von Eingabe- und Ausgabeumleitung, Lesen aus
\texttt{adressbuch}, Schreiben in \texttt{sortiert} 

\subsubsection{Standardfehlerausgabe}
\texttt{cat /etc/shadow 2> fehlerfile} \\ 
Die Fehlerausgabe des Programmes \texttt{cat} wird in eine Datei namens
\texttt{fehlerfile} umgeleitet. 

\texttt{find / -name adressbuch > gefunden 2> fehlerfile} \\
Die Ausgabe des Programmes \texttt{find} wird in eine Datei namens
\texttt{gefunden} umgeleitet, die Fehlermeldungen (z.B. Zugriff nicht
erlaubt) landen in einem \texttt{fehlerfile}.

\texttt{find / -name adressbuch \&> resultat} \\ 
Alles, was \texttt{find} ausgibt, wird in eine Datei namens
\texttt{resultat} umgeleitet - auch die Fehlermeldungen.

\subsubsection{Pipe}
Alle bisher angeführten Umleitungszeichen stellen immer die Verbindung
eines Programmes mit einer Datei her. Um die Ausgabe eines Programmes
der Eingabe eines anderen Programmes zuzuführen, verwenden wir die
Pipe:

\texttt{cat adressbuch | grep BULME} \\
Die Datei \texttt{adressbuch} wird mittels des Programmes \texttt{cat
} ausgegeben; die Ausgabe wird dem Programm \texttt{grep} übergeben, das
Zeilen ausgibt, die die Zeichenkette \texttt{BULME} enthalten.

\texttt{date | tee erstellungsdatum} \\
Die Ausgabe des Programmes ''date'' wird an ein ''T-Stück''
weitergeleitet, das die Daten gleichzeitig an der Standardausgabe
anzeigt und in die Datei ''erstellungsdatum'' schreibt.
 
\subsubsection{HERE-Dokumente}
Bei diesem Sonderfall der Eingabeumleitung kommen die Eingaben direkt
aus dem Shellskript. Der Text wird dabei von einer beliebig wählbaren
Zeichenfolge umgeben:
\begin{verbatim}
#! /bin/bash
tr [A-Z] [a-z] << TEXT
null
eins zwei drei
vier 
TEXT
\end{verbatim}

\section{Wo finde ich Hilfe?}
\subsection{''Eingebaute'' Hilfe}
Manche Programme haben eine eigene Option \texttt{--help} zur Ausgabe
einer Kurzhilfe; darüberhinaus wird gelegentlich auch bei
falschen/fehlenden Argumenten oder Optionen eine Kurzanleitung
angezeigt.

\subsection{Manual Pages}
Das wichtigste Hilfesystem unter Unix, Aufruf mit: \\
 \texttt{man [ABSCHNITT] [OPTIONEN] SUCHWORT}.\\

\texttt{SUCHWORT} muss auf jeden Fall angegeben werden, bei fehlender
\texttt{ABSCHNITT} wird jene mit der niedrigsten Nummer angezeigt.
Manual-Pages haben eine einheitliche Syntax, Details unter \texttt{man
man}. Kommandos, Argumente und Optionen werden folgendermaßen gekennzeichnet:

\begin{tabular}{ll}
\textbf{text} & text wörtlich wie in der Anzeige \\
$ \underline{text}$ & text ersetzen durch ein passendes Argument \\
$[$\textbf{-a}$]$ & a ist ein optionales Argument \\
$[$\textbf{-a$|$b}$]$ & es kann nur entweder Argument a oder Argument b benützt
werden \\
$ \underline{text}$... &  mehrere Optionen oder Argumente möglich\\
\end{tabular} \\

Weiters liefert \texttt{whatis SUCHWORT} eine einzeilige
Kurzbeschreibung eines Befehls.  Mit \texttt{apropos SUCHWORT} werden
die Kurzbeschreibungen aller Manual-Pages nach \texttt{SUCHWORT}
durchforstet.

\subsection{Weitere Hilfe}
Unter Linux sind in der Verzeichnishierarchie unter /usr/share/doc
teilweise recht ausführliche Hilfedateien vorhanden. 

\subsection{Online-Quellen}
\begin{itemize}
\item Linux Documentation Project http://tldp.org/ mit HOWTOs, FAQs und kompletten Büchern

\item https://tldr.inbrowser.app/
\item https://cheat.sh (Kommando mit / anhängen, geht auch im Kommandofenster beispielsweise \texttt{curl cheat.sh/ls})
\item https://overthewire.org/wargames/ (Lernen der shell kann auch Spaß bereiten)
\end{itemize}


\section{Dateien und Verzeichnisse}
\subsection{Der Verzeichnisbaum eines Linux-Systems}
\begin{samepage}
\begin{verbatim}
/
|-- bin
|-- boot
|-- cdrom
|-- data
|-- dev
|-- etc
|-- floppy
|-- home
|-- initrd
|-- lib
|-- lost+found
|-- mnt
|-- opt
|-- proc
|-- root
|-- sbin
|-- sys
|-- tmp
|-- usr
|-- var

\end{verbatim}
\end{samepage}

\subsection{Navigieren im Dateisystem}
Einige Verzeichnisse sind besonders gekennzeichnet:\\

\begin{tabular}{ll}
. & das aktuelle Verzeichnis \\
.. & das dem aktuellen Verzeichnis übergeordnete Verzeichnis \\
$\sim$ & das eigene home-Verzeichis \\
$\sim$\texttt{fred} & Kurzschreibweise für \texttt{/home/fred} \\
\end{tabular} \\

\texttt{cd} (ohne Argumente) wechselt immer ins home-Verzeichnis des
Benutzers, unter dem man gerade eingeloggt ist.

Pfadangaben können auf zwei Arten erfolgen:
\begin{itemize}
\item absolut (ausgehend vom Wurzelverzeichnis /), z.B: \\
\texttt{/home/fred/bilder/ferien04/girl.jpg}
\item relativ (vom aktuellen Verzeichnis aus):\\
\texttt{bilder/ferien04/girl.jpg}\\
oder zur Verdeutlichung:\\
\texttt{./bilder/ferien04/girl.jpg}
\end{itemize}

\subsection{Dateitypen}
Unter Unix ist beinahe alles eine Datei, auch Systemressourcen wie
Speichermedien, Netzwerklaufwerke, Ethernet-Adapter und Mäuse. Der
Dateityp wird durch bestimmte Flags, aber nicht wie unter Windows durch
die Erweiterung bestimmt, d.h. ein ausführbares Programm hat nicht
notwendigerweise die Form \texttt{programm.exe}. Wenn ein Datei eine
bestimmte Erweiterung hat, so wird diese vom Anwendungsprogramm (z.B.
OpenOffice) verlangt und nicht vom Betriebssystem. Herausfinden lässt
sich der Dateityp durch Eingabe des Befehls:\\
\texttt{ls -l DATEI} \\
(Dateityp steht im ersten Feld der Ausgabezeile) oder durch \\
\texttt{file DATEI}. 


Die wichtigsten Dateitypen sind (Bsp. jeweils mit der Ausgabe von
\texttt{ls -ld}):
\begin{itemize}
\item Gewöhnliche Dateien (Texte, Daten, Konfigurationsdateien):\\
\texttt{-rw-r--r--    1 root     root          905 2004-07-07 14:05
/etc/passwd}
\item ausführbare Dateien (Programe): \\
\texttt{-rwxr-xr-x    1 root     root       142896 2003-09-23 19:33
/bin/tar}
\item Verzeichnisse (Directories): \\
\texttt{drwxr-xr-x   55 root     root         5728 2004-10-04 14:20
/etc}
\item Verweise (Hard-, Softlinks): \\
\texttt{lrwxrwxrwx    1 root     root           10 2004-07-07 13:40 mail
-> spool/mail}
\item Gerätedateien (Block-, Character-Devices): \\
\texttt{brw-rw----    1 root     disk       3,   0 2003-09-23 19:59
/dev/hda} \\
\texttt{crw-rw----    1 root     uucp       5,  64 2003-09-23 19:59
/dev/cua0}
\item Fifo (named pipe): \\
\texttt{prw-r--r--    1 edgar    users           0 2004-10-04 18:06 xx}
\item Socket (Netzwerk-Endpunkt): \\
\texttt{srw-rw-rw- 1 root root 0 Sep 24 07:40 /run/cups/cups.sock}
\end{itemize}


\subsection{Zugriffsrechte}
Unter Unix gibt es die Möglichkeit, Lese-, Schreib- und
Ausführungsrechte an den Eigentümer, die Gruppe zu der er gehört und
für alle anderen Benutzer zu vergeben. Diese Rechte lassen sich mit
wiederum mit \texttt{ls -l} anzeigen, mit der Ausgabe: \\

\texttt{-rwxr-xr-x    1 kurs     users           8 2004-10-06 15:50 testfile } \\

Nach dem ersten Ausgabefeld (-) kommen drei Dreiergruppen von Zeichen
nach dem Schema:

\begin{tabular}{|l|c|c|c|}
\hline
Rechte für: & (u)ser & (g)roup & (o)ther \\
\hline
\hline
mit Buchstaben: & rwx & r-x & r-x \\ 
binär: & 111 & 101 & 101  \\
oktal: & 7 & 5 & 5 \\
\hline
\end{tabular}\\

Eigentümer der Datei ist \texttt{kurs}, die Gruppe mit Zugriffsrechten
heißt in diesem Beispiel \texttt{users}.

Gesetzt werden die Dateirechte mit:
\texttt{chmod [OPTION]... MODUS[,MODUS]... DATEI...}

oder \texttt{chmod [OPTION]... OKTAL-MODUS DATEI...}\\

\texttt{MODUS} besteht aus 3 Teilen:
\begin{itemize}
\item Wer: u (user, Eigentümer), g (group), o (alle anderen), a (alle,
  d.h Eigentümer, Gruppe und alle anderen)
\item Operator: + (Rechte geben), - (Rechte wegnehmen), = (Rechte
  setzen)
\item Rechte: r (read, lesen), w (write, schreiben), x (execute,
  ausführen, bei Verzeichnissen durchsuchen). Spezialflags s (setuid oder setgid), t (sticky) siehe Erklärung weiter unten.
\end{itemize}

In der zweiten Form wird der \texttt{OKTAL-MODUS} wie in der Tabelle
angedeutet durch drei Oktalziffern gesetzt. 

Die Zuordnung von Dateien zu einem Eigentümer erfolgt durch: \\
\texttt{chown EIGENTÜMER DATEI...}

Zuordnung zu einer Gruppe mit:\\
\texttt{chgrp GRUPPE DATEI...}

Eine ganze Verzeichnishierarchie wird mit der Option \texttt{-R} (rekursiv)
bearbeitet.

Zusätzlich gibt es noch 3 Spezialbits, die optional der Dreiergruppe vorangestellt werden können, in Oktalzahlen ausgedrückt bedeuten sie:

\texttt{4000} $ \rightarrow $ setuid: Datei wird mit den Rechten des Besitzers ausgeführt \\
Beispiel: \texttt{-rwsr-xr-x 1 root root 59976 Feb  6  2024 /bin/passwd} \\
Achtung: wenn der Besitzer \texttt{root} heißt, kann bei falscher Anwendung oder Programmierfehlern eine gefährliche Sicherheitslücke entstehen!

\texttt{2000} $ \rightarrow $ setgid: Datei erhält die Rechte der Gruppe \\

\texttt{1000} $ \rightarrow $ sticky: nur Eigentümer darf Datei löschen \\
Beispiel: \texttt{drwxrwxrwt 33 root root 20480 Sep 24 14:59 /tmp} \\

Welche Rechte eine Datei standardmäßig hat, lässt sich in der
Benutzerkonfiguration einstellen durch:
\texttt{umask MODE}

\section{Suchen, finden und ersetzen}
\subsection{Suchen von Dateien und Verzeichnissen im Dateisystem}
\texttt{find [Pfad...] [Suchkriterium]}\\

Die wichtigsten Spezialfälle (Details siehe Manual Page) anhand
von Beispielen:\\

\texttt{find / -name fred} \\
Vom Wurzelverzeichnis beginnend alle Dateien und Verzeichnisse
anzeigen, die \texttt{fred} heißen. Hinweis: Bei Wildcards (*, ?, [ ])
im Namen müssen Quotes '' '' verwendet werden.

\texttt{find / -user kurs} \\
Vom Wurzelverzeichnis beginnend alle Dateien und Verzeichnisse
anzeigen, die dem Benutzer \texttt{kurs} gehören.

\texttt{find /tmp -mtime +3 -print -exec rm \{\} $\backslash$;} \\
Alle Dateien und Verzeichnisse unter dem Verzeichnis \texttt{/tmp}
ausgeben, falls sie länger als 3 Tage nicht modifiziert wurden.
Alles was nach \texttt{-exec} steht, ist der Aufruf eines weiteren
Programmes (hier \texttt{rm}, löschen). Mit \{\} werden dem Programm
die von \texttt{find} ausgegebenen Zeilen als Argumente übergeben. Die
Kommandozeile von \texttt{rm} wird mit ; abgeschlossen, wobei der
Backslash verhindert, dass die Shell den Strichpunkt schon vorher
interpretiert (Quoting).

\texttt{find $\sim$karl -type d -exec chmod 775 \{\} $\backslash$;} \\
Alle Verzeichnisse unterhalb des Heimverzeichnisses vom Benutzer
\texttt{karl} werden auf \texttt{rwxrwxr-x} gesetzt.

\texttt{find . -mtime -4 -a -mtime +1} \\
Finde alle Dateien, die vor weniger als 4 Tagen  und vor mehr als 1 Tag geändert wurden. Die Option \texttt{-a} repräsentiert ein logisches AND, fehlt sie, wird automatisch diese Verknüpfung angenommen.


\subsection{Suchen/Ersetzen von Text innerhalb von Dateien} 
\subsubsection{Reguläre Ausdrücke (regular expressions)}
\begin{tabular}{|l|l|}
\hline
regulärer Ausdruck: & Bedeutung:\\
\hline
\texttt{a+} & a kommt $\ge$ 1mal vor \\
\texttt{a*} & a kommt $\ge$ 0mal vor \\
\texttt{a?} & a kommt $\le$ 1mal vor \\
\texttt{a\{,n\}} & a kommt maximal n-mal vor \\
\texttt{a\{n,\}} & a kommt mindestens n-mal vor \\
\verb1^a1 & a am Zeilenanfang \\
\texttt{a\$} & a am Zeilenende \\
\texttt{$\backslash$<} & Beginn einer Wortgrenze \\
\texttt{$\backslash$>} & Ende einer Wortgrenze \\
\texttt{.} & ein beliebiges Zeichen \\
\texttt{[a-z]} & eines der Zeichen von a bis z \\
\texttt{[xyz]} & eines der Zeichen x,y,z \\
\verb1[^xyz]1 & keines der Zeichen x,y,z (Negation)\\
\hline
\end{tabular}\\

Die Wildcards der shell und reguläre Ausdrücke  sehen ähnlich aus,
sind aber zwei unterschiedliche Dinge! Während Wildcards von der shell
zu Dateinamen erweitert werden, finden reguläre Ausdrücke als Argumente
zum Suchen und Ersetzen von Zeichenketten in Programmen wie grep, vi, sed,
awk Verwendung.\\  


Beispiele:\\
\begin{tabular}{ll}
Muster & Passt auf:\\
\hline
\texttt{Code} & Die Zeichenkette \texttt{Code}\\
\verb1^Code1 & Die Zeichenkette \texttt{Code} am Zeilenanfang\\
\texttt{Code\$} & Die Zeichenkette \texttt{Code} am Zeilenende\\
\verb1^Code$1 & Die Zeichenkette \texttt{Code} steht allein in der Zeile\\
\texttt{[CK]ode} & \texttt{Code} oder \texttt{Kode}\\
\texttt{Co.e} & Der dritte Buchstabe ist ein beliebiges Zeichen\\
\verb1^....$1 & Jede Zeile mit genau 4 Zeichen\\
\verb1^\.1 & Jede Zeile die mit einem Punkt beginnt (Metazeichen
gequotet)\\
\verb1^[^.]1 & Jede Zeile die nicht mit einem Punkt beginnt\\
\texttt{Code*} & \texttt{Code Codewort Codeknacker Codeschloss} usw.\\
\texttt{[0-9]*} & Null oder mehrere Ziffern\\
\texttt{[0-9]+} & Eine oder mehrere Ziffern\\
\texttt{[0-9].*} & Eine Ziffer, gefolgt von null oder mehr Zeichen \\
\texttt{80[456]?86*} & \texttt{8086 80486 80586 80686} \\
\texttt{Hand(y|ie)s} & \texttt{Handys Handies} \\
\end{tabular}


\subsubsection{Suchen mit ''grep''}

\texttt{grep [OPTION]... MUSTER [DATEI] ...} \\
Als \texttt{MUSTER} dienen sogenannte ''reguläre Ausdrücke''
Falls im \texttt{MUSTER} Sonderzeichen bzw. Wildcards vorkommen,
müssen sie unter Hochkommata gestellt werden (quoting, z.B: \texttt{'Hand(y|ie)s'}), da sie ansonsten
bereits von der shell und nicht erst von grep ausgewertet werden. 

\textit{Hinweis: einige Ausdrücke (oben beispielsweise die letzten 5 Muster) funktionieren nur in der erweiterten
Version \texttt{grep -E}}

Beispiel:\\
In einem Verzeichnis befinden sich die Dateien\\ 
\texttt{array.c band.c kap1 kap2}

Zur Suche nach einem Muster aus beliebigen Kleinbuchstaben in den
Dateien \texttt{kap1 kap2} wird folgende Zeile eingegeben:\\
\texttt{grep [a-z]* kap[12]}

Das wird aber von der shell interpretiert als\\
\texttt{grep array.c band.c kap1 kap2} \\
während \texttt{[a-z]*} eigentlich nur das an \texttt{grep} übergebene
Suchmuster sein sollte. Um die Interpretation der Metazeichen im
Suchmuster durch grep zu erzwingen, muss daher\\
\texttt{grep '[a-z]*' kap[12]}\\
eingegeben werden.\\

\subsubsection{Suchen und ersetzen mit dem Stream-Editor ``sed''}

\texttt{sed [-e 'Befehle'] [ DATEI]} \\
Wird keine Datei angegeben, liest sed aus der Standardeingabe.\\

Beispiel: Jedes Mal, wenn im File ``hallo.txt'' das Wort ``Windows''  vorkommt,
soll dieses durch ``Linux'' ersetzt werden. (Der Schalter ``g'' für ``global''
bewirkt, dass das Wort auch ersetzt wird, wenn es in einer Zeile mehrmals vorkommt).
\begin{verbatim}
sed -e 's/Windows/Linux/' hallo.txt
\end{verbatim}

Beispiel: Leerzeilen aus einer Datei löschen (d = ''delete''):
\begin{verbatim}
sed -e '/^$/d' hallo.txt
\end{verbatim}

Beispiel: Leer- und Kommentarzeilen aus einer Datei löschen, das Zeichen \texttt{|} 
bedeutet ODER-Verknüpfung. Da es nicht Bestandteil des Suchmusters ist, muss ein 
Backslash vorangestellt werden:
\begin{verbatim}
sed -e '/^$\|#/d'
\end{verbatim}

Backups anlegen:
\begin{verbatim}
#!/bin/sh
EXTENSION="txt"
for name in *.$EXTENSION ; do
        backup=$(echo $name | sed -e "s/\.$EXTENSION/\.bak/")
        echo "$name -> $backup"
        cp $name $backup
done
\end{verbatim}

In der Option nach dem \texttt{sed} - Kommando müssen doppelte Anführungszeichen gesetzt werden, damit der Inhalt der Variablen \texttt{EXTENSION} ausgegeben wird.

Der Ausdruck \texttt{backup=\$(......)} weist der Variable backup den Inhalt der Ausgabe des Kommandos innerhalb der Klammern zu.



\section{Prozesse}
\subsection{Welcher Prozesse laufen?}
Beim Start eines Programmes wird aus einem existierenden Prozess ein
neuer erzeugt (fork). Der ''Vaterprozess'', der beim Hochfahren
gestartet wird, heißt init.Mit \texttt{pstree} lässt sich diese
Hierarchie visualisieren, wie der folgenden Ausschnitt zeigt:
\begin{samepage}
\begin{verbatim}
init-+-acpid
     |-bdflush
     |-cron
     |-cupsd
     |-fetchmail
     |-kalarmd
     |-kamix
     |-kdeinit-+-artsd
     |         |-3*[kdeinit]
     |         |-kdeinit---bash-+-xdvi.bin
     |         |                |-xemacs
     |         |-kdeinit---bash---ssh
     |         |-kdeinit---bash---pstree
\end{verbatim}
\end{samepage}

Laufende Prozesse werden angezeigt mit: 
\texttt{ps}\\
(Ausgabeformat: \texttt{PID}...Prozessnummer, \texttt{CMD}...Programmname)

Prozesse, die nicht aus einem Terminalfenster heraus gestartet werden
(sondern z.B. aus dem KDE-Startmenü), erscheinen erst bei Eingabe von:
\texttt{ps x}\\
\textit{(Hinweis: die Optionen werden hier ohne führendes - angegeben,
  da ps die BSD-Syntax verwendet)}.

Eine tabellarische Darstellung, die auch Interaktionen des Benutzers
ermöglicht, liefert:
\texttt{top} 
(Hilfe mit h, beenden mit q).


\subsection{Hintergrund-Prozesse}
Wird ein Prozess im Hintergrund gestartet, z.B. mit:
\texttt{xeyes \&} \\
so erscheint in der folgenden Zeile z.B: 
\texttt{[2] 2201} (Format: [JOBNUMMER] PID)

Der Prozess wird in den Vordergrund geholt mit:
\texttt{fg \%JOBNUMMER}

zurück in den Hintergrund geht es mit:
\keys{CTRL}+\keys{Z}
\texttt{bg} 

Die Jobnummern der laufenden Hintergrundprozesse liefert:
\texttt{jobs}

\subsection{Prozesse beenden}
Beenden eines nicht reagierenden Programmes:
\texttt{kill PROZESSNUMMER}\\
Bei Hintergrundprozessen alternativ auch:
\texttt{kill \%JOBNUMMER}

Beenden aller Programme mit Namen PROGRAMMNAME:
\texttt{killall PROGRAMMNAME}.

Mit \texttt{kill} können als Optionen auch Signale mitgesendet werden.
Gewaltsames Beenden eines nicht mehr reagierenden Programmes (SIGKILL):
\texttt{kill -9 PROZESSNUMMER}

Beenden eines Programmes mit anschließendem Neustart (z.B. zum Einlesen neuer
Konfigurationsdaten):
\texttt{kill -HUP PROGRAMMNAME} 

\section{Tools}
\subsection{Editoren}
\texttt{vi} oder die verbesserte Version \texttt{vim} ist der auf jedem Unix-System vorhandene Standard-Editor. Er ist schnell, klein und benutzerfreundlich auf eine besondere Art - er
kennt zwei Betriebsarten, den Kommandomodus (Zustand nach dem Start,
Befehlseingabe) und den Eingabemodus (insert mode, Texteingabe).\\

\begin{tabular}{|l|l|}
\hline
\texttt{i} & Wechsel Kommandomodus $\rightarrow$ Eingabemodus mit i \\
\keys{\texttt{ESC}} & Wechsel Eingabemodus  $\rightarrow$ Kommandomodus \\
\texttt{o} & neue Zeile anfügen \\
\texttt{dd} & aktuelle Zeile löschen (delete)\\
\texttt{yy} & aktuelle Zeile kopieren (yank)\\
\texttt{p} & gelöschte oder kopierte Zeile einfügen (put)\\
\texttt{x} & ein Zeichen löschen \\
\texttt{u} & Änderungen rückgängig machen (undo)\\
\texttt{:w} & speichern (write)\\
\texttt{:wq} & speichern und beenden (quit)\\
\texttt{ZZ} & \\
\texttt{:q!} & beenden, ohne Änderungen zu speichern\\
\texttt{:\%s/alt/neu/g} & überall \texttt{alt} durch \texttt{neu} ersetzen\\
\texttt{help} & Onlinehilfe \\
\hline
\end{tabular}
\\

Ein weiterer, in LISP frei programmierbarer Editor ist emacs. Er
arbeitet mit vielen gängigen Unix-Werkzeugen wie z.B. grep, gcc, make
zusammen. Neuere Entwicklungen in Richtung integrierte Designumgebung
(IDE) sind visual studio code, kdevelop (Bestandteil der Desktop-Umgebung KDE) und eclipse
(Java-basiert, mit plugins erweiterbar).

\subsection{Compiler}
Der gcc (GNU compiler collection) zum Kompilieren von C-Programmen
wird so verwendet:
\texttt{gcc hello.c} \\
Das resultierende ausführbare Programm heißt traditionsgemäß \texttt{a.out}.

\texttt{gcc hello.c -o hello} 
liefert ein Programm namens \texttt{hello}\\
\textit{(Hinweis: Der Kompiler setzt die Rechte der erzeugten
  Programme automatisch auf e(x)ecutable. Dennoch liefert die Eingabe
  von hello auf den meisten Systemen ein ''command not found'': Der
  Suchpfad \$PATH zum Programm ist aus Sicherheitsgründen nicht auf
  das aktuelle Verzeichnis gesetzt, daher ist eine Eingabe der Form
  ./hello notwendig)} 

Nur kompilieren ohne zu linken:
\texttt{gcc -c hello.c}

Das entstehende Object File \texttt{hello.o} wird gelinkt mit:
\texttt{gcc -c hello.o}

\subsection{Archivierung mit tar (''tape archiver'')}
\texttt{tar [OPTION]... [Datei]...}\\
\textit{OPTION ist in der Unix- und in der BSD-Form (ohne -) möglich.
  Bei der BSD-Form ist die Reihenfolge der Optionen egal!}

Erzeugen eines komprimierten Archivs (auch mit winzip extrahierbar!):\\
\texttt{tar zcvf freds-backup.tgz /home/fred}

Auslesen des Inhalts dieses Archivs:\\
\texttt{tar zvft freds-backup.tgz}

Entpacken:\\
\texttt{tar zvfx freds-backup.tgz}

Bedeutung der Optionen:\\
\begin{tabular}{|ll|}
\hline
\texttt{v} & ausführliches Listing (verbose) \\
\texttt{z} & (de)komprimieren mit gz (mit j anstelle von z: bzip2)\\
\texttt{c} & Archiv erzeugen (create) \\
\texttt{v} & ausführliches Listing (verbose) \\
\texttt{f} & Archiv ist ein (f)ile, ohne die Option Schreiben auf ein tape \\
\texttt{t} & Inhalt auflisten ohne zu entpacken \\
\texttt{x} & e(x)trahieren \\
\hline
\end{tabular}

\section{Shellskripts} 
\subsection{Begriffsdefinition und Einsatzgebiete}
Zur automatischen Abarbeitung längerer oder komplexer Befehlsfolgen
bietet sich die Verwendung von Skripts an. Routinemäßige
Administrationsaufgaben können so vereinfacht werden oder überhaupt
als cron-job periodisch gestartet werden. Auch das Hochfahren des
gesamten Linux-Systems wird durch Shellskripts gesteuert.  

Im Gegensatz zu kompilierten Programmen werden Skripts interpretiert,
die Ausführung ist daher etwas langsamer; andererseits ist die
Erzeugung von Scripts unkomplizierter (Compileraufruf entfällt). Oft
sind Skripts auch leichter auf andere Plattformen (z.B. Windows,
MaxOS) portierbar. Als Interpreter dient im einfachsten Fall die Shell
selbst (bash, sh, csh usw.) - das Pendent dazu aus der DOS-Welt heißt
Batchfile. Für spezielle Anforderungen oder größere Projekte stehen
Skriptsprachen wie perl, pyton, tcl etc. zur Verfügung.

\subsection{Systemumgebung}
Grundvoraussetzungen zum Ausführen eines Shellskripts:
\begin{itemize}
\item Der entsprechende Interpreter muss auf dem System installiert sein.
\item Ganz am Beginn des Skripts sollte der Interpreter des Skriptes
  angeführt werden. Falls ein Benutzer z.B. seine Arbeitsumgebung so
  konfiguriert hat, dass standardmäßig die csh läuft, wird auf diese
  Weise die Verwendung der bash sichergestellt:\\
  \texttt{\#! /bin/bash}\\
  Das gleiche für Python: \\
  \texttt{\#! /usr/bin/python}\\
  oder etwas etwas systemunabhängiger geschrieben, wobei berücksichtigt wird, dass Pfad zum Interpreter von   System zu System variieren kann: \\
  \texttt{\#! /usr/bin/env python}\\
\item Der Benutzer muss das Skript ausführen können. Unter Unix
  geschieht das nicht durch Erkennen einer Dateinamen-Erweiterung wie
  .bat oder .exe unter DOS (d.h. die Namenswahl ist weitgehend frei)
  sondern durch Setzen der Rechte, z.B. mit: \\
  \texttt{chmod u+xmeinskript}
\item Das Verzeichnis, in dem das Skript zu finden ist, muss im
  Suchpfad eingetragen sein - siehe: 
  \texttt{echo \$PATH}\\
  Ist das nicht der Fall, so ist die Angabe des absoluten oder
  relativen Pfades zur Ausführung nötig:
  \texttt{/home/fred/meinskript}
  oder:
  \texttt{./meinskript}
\end{itemize}

Achtung: Shellskripts sind kritisch bezüglich der Formatierung (Leerzeichen, Zeilenumbruch)!

\subsection{Shellvariablen}
Kennzeichen von interpretierten Sprachen ist u.a. die Speicherung von
Variablen in Form von Zeichenketten. 

\subsubsection{Definition, Ausgabe, Löschen}
\begin{itemize}
\item Lokale Variablen werden so definiert:\\
\texttt{x=100}\\
\texttt{mytext="Wien ist anders"} (Quoting wegen Leerzeichen!)\\
\texttt{pi=3.14}

\item Globale Variablen (Umgebungsvariablen, environment variables) werden
an Kindprozesse weitergegeben. Lokale werden zu globalen Variablen
mit:
\texttt{export variable}\\
oder gleich bei der Definition mit
\texttt{export variable=100}

\item Anzeigen des Wertes einer Variablen mit:
\texttt{echo \$variable}\\
Die ausführlichere (und sichere) Schreibweise dafür ist: 
\texttt{echo \$\{variable\}}\\

Beispiel:\\
\texttt{gruss="hallo "}\\
\texttt{echo \$grussleute} ...ist leer, weil andere Variable\\
\texttt{echo \$\{gruss\}leute} ... ergibt ``hallo leute'' 

\item Anzeigen aller globalen Variablen:
\texttt{printenv}

\item Anzeigen aller (globalen und lokalen) Variablen:
\texttt{set}

\item Löschen einer Variable x:
\texttt{unset x}
\end{itemize}

\subsubsection{Einige reservierte Shellvariablen}
\begin{tabular}{|l|l|}
\hline
\texttt{\$?} & Rückgabewert (return) des letzten Kommandos\\
\texttt{\$!} & PID des letzten Hintergrundkommandos\\
\texttt{\$\$} & PID der aktuellen Shell\\
\texttt{\$0} & Name des gerade ausgeführten Shell-Programmes\\
\texttt{\$\#} & Anzahl der übergebenen Parameter (in C: argc)\\
\texttt{\$1...\$9} & Werte der übergebenen Parameter (in C: argv)\\
\texttt{\$?} & Rückgabewert (return) des letzten Kommandos\\
\texttt{\$*} & Übergebene Parameter als ein String\\
\texttt{\$\makeatletter @} & Übergebene Parameter als ein Array\\
\hline
\end{tabular} 

Vom System in der Konfiguration gesetzte Variablen sind z.B
\texttt{\$PATH, \$UID, \$PS1, \$PWD, \$IFS}.

\subsubsection{Einlesen von Kommandozeilenparametern aus der Konsole}
Die beim Aufruf eines Skripts angegebenen Optionen werden der Reihe nach
den Variablen \$1, \$2 usw. zugeordnet. Alternativ dazu kann mit ''shift''
die Parameterliste verschoben werden, sodass die jeweils nächste Option
immer in \$1 steht:
\begin{verbatim}
#!/bin/sh
echo "Alle Parameter:"
count=1
while [ $# -gt 0 ] ; do
        echo "${count}. Parameter = $1"
        shift
        count=$((count+1))
done
\end{verbatim}

\subsubsection{Einlesen mit read}
\begin{itemize}
\item Einlesen mehrerer Variablen von der Konsole:
\texttt{read vorname familienname}\\
...und die dazugehörige Ausgabe mit:
\texttt{echo \$vorname \$familienname}

\item Das gleiche unter Verwendung eines Arrays:
\texttt{read -a namen}\\
...und die Ausgabe:
\texttt{echo \$\{namen[0]\} \$\{namen[1]\}}

\item Einlesen aus einer Datei, in der die Felder durch '':'' getrennt
sind unter Verwendung des IFS (internal field separator):
\begin{verbatim}
#!/bin/sh
IFS=":"
while read -a fields
do
        echo ${fields[@]}
done < /etc/passwd
\end{verbatim}

Wenn read mit einer Pipe verwendet wird, öffnet es eine eigene Subshell.
Daher muss auch der Ausgabebefehl durch Klammerung in die Subshell
miteinbezogen werden - andernfalls wären die Variablen leer:
\begin{verbatim}
#!/bin/sh
ADDR=$(/sbin/ifconfig | grep eth -n1 | grep inet | cut -d: -f2 | \
     cut -d" " -f1)
echo $ADDR |  (IFS="."; read -a felder; echo ${felder[@]})
\end{verbatim}
\end{itemize}

\subsubsection{Parametersubstitution mit Beispielen}
\texttt{pfad="/home/fred/datei.txt.gz"} ...Variable pfad wird
gesetzt

\texttt{echo \$\{pfad\#*/\}}\\
liefert home/fred/datei.txt.gz (Führender Schrägstrich entfernt)

\texttt{echo \$\{pfad\#\#*/\}}\\
liefert datei.txt.gz - gleiche Funktionalität wie 
\texttt{basename \$pfad}

\texttt{echo \$\{pfad\%/*\}}\\
liefert /home/fred

\texttt{echo \$\{pfad\%\%/*\}}\\
eliminiert die größtmögliche Zeichenkette vom Ende weg, also alles

\texttt{echo \$\{pfad:-nix\}}\\
liefert /home/fred/datei.txt.gz\\
\texttt{unset pfad} ...Variable ist jetzt leer\\
\texttt{echo \$\{pfad:-nix\}}\\
liefert ``nix''


\subsection{Kontrollstrukturen/Verzweigungen}
\subsubsection{if}
\begin{verbatim}
if [ $x -ne 3 ] ; then
   echo "falscher Wert"
   exit 1
fi
\end{verbatim}

Der Strichpunkt kann wegfallen, wenn \texttt{then} in der nächsten Zeile steht:
\begin{verbatim}
if [ $x -ne 3 ] 
then
   echo "falscher Wert"
   exit 1
fi
\end{verbatim}

Die eckigen Klammern sind eine abgekürzte Schreibweise für:

\begin{verbatim}
if test $x -ne 3  ; then
   echo "falscher Wert"
   exit 1
fi
\end{verbatim}

\textit{Hinweis: Vergleichsoperatoren siehe ``man test''}

\subsubsection{case}
\begin{verbatim}
case $eingabe in
   s)
      echo Start
      ;;
   [eqEQ])
      echo Ende
      exit 0
      ;;
   *)
      echo "falsche Eingabe!"
      ;;
esac
\end{verbatim}
Sobald eine Alternative zutrifft, wird die case-Struktur verlassen (im
Gegensatz zur Programmiersprache ''C'', wo dieses Verhalten durch ein
explizites ''break'' erzwungen werden muss).


\subsubsection{Schleifen: while, until,for}
\begin{verbatim}
counter=0
while [ $counter -lt 3 ] ; do
   echo $counter
   counter=$[$counter+1]
done
\end{verbatim}

Negative Formulierung mit until:
\begin{verbatim}
counter=0
until [ $counter -gt 3 ] ; do
   echo $counter
   counter=$[$counter+1]
done
\end{verbatim}


Iteration über Werte aus einer Liste:
\begin{verbatim}
for opsys in Windows Linux MacOS; do
   echo $opsys 
done
\end{verbatim}

Dateien umbenennen (Einzeiler): 
\begin{verbatim}
for $filename in *.doc; do mv $filename ${filename%.*}.bak; done
\end{verbatim}
\textit{Auf diese Weise kann ein Skript mit der selben Funktion wie
  der DOS-Befehl move *.doc *.bak realisiert werden. (mv *.doc *.bak
  funktioniert unter Unix nicht, weil die Shell und nicht das Programm
  mv die Wildcards erweitert. Mit unkritischen Dateien ausprobieren!)}

Ohne \texttt{in} liest \texttt{for} die Argumente der Eingabezeile.Mit
\texttt{break} wird aus der Schleife ausgestiegen, mit
\texttt{continue} geht es direkt weiter zur nächsten Iteration.

\subsection{Rückgabewerte} 
Ordentliche Unix-Programme geben bei erfolgreicher Beendigung 0
(Null), bei einem Fehler einen von 0 verschiedenen Wert (Fehlercode)
zurück. Aufrufende Programme können diesen Rückgabewert auswerten und
darauf reagieren. Beim Shellskript erzeugt man diese Werte mit:\\
\texttt{exit 0} bzw. \texttt{exit 1}

\subsection{Verkettung von Kommandos}
\begin{itemize}
\item{Hintereinanderausführen}\\
Durch Strichpunkte '';'' getrennte Befehle werden nacheinander
ausgeführt:\\
\texttt{date; who}\\
Um die Ausgabe beider Befehle in eine Datei umzuleiten, wird durch
Klammern die Ausführung in einer Subshell erzwungen:\\ 
\texttt{(date; who) >  heute\_da}\\

\item{UND-Verknüpfung}\\
Das nächste Kommando wird nur ausgeführt, wenn das vorherige erfolgreich war.\\
Beispiel: Konfiguration, Kompilieren und Installation eines im Quellcode 
vorliegenden Programmes.\\
\texttt{./configure \&\& make \&\& make install}\\

\item{EXOR-Verknüpfung}\\
Das nächste Kommando wird nur ausgeführt, wenn das vorherige erfolglos war.\\
\texttt{grep Franz namensliste  || echo ''Franz nicht vorhanden''}\\
\end{itemize}

\subsection{Rechnen mit der Shell}
Die Bash erlaubt das Rechnen mit allen Grundrechenarten (+,-,*,/)
einschließlich des Modulo-Operators (\%) sowie mit Inkrement und 
Dekrement (++, --). Ebenfalls erlaubt sind Vergleichsoperatoren wie in C. Zur
Auswertung eines arithmetischen Ausdruckes existieren mehrere
Syntaxvarianten, am einfachsten und sichersten in Bezug auf die 
Behandlung von Sonderzeichen ist die doppelte Klammerung:

\begin{verbatim}
#!/bin/sh
echo "Alle Parameter:"
count=1
while [ $# -gt 0 ] ; do
	echo "${count}. Parameter = $1"
	shift
	count=$((count+1))
done
\end{verbatim}

\subsection{Geordneter Rückzug mit ''trap''}
Beim gewaltsamen Beenden von Skripts, z.B. mit ''kill'' oder 
\keys{\texttt{Ctrl}}-C können eventuell notwendige Aufräumarbeiten am
Ende des Skripts nicht mehr stattfinden, sodass temporäre Files etc. 
übrigbleiben. Mit ''trap'' lassen sich Aktionen definieren, die beim Abbruch
ausgeführt werden. \texttt{trap -l} zeigt alle Signale und deren Nummern an.
\begin{verbatim}
#!/bin/sh

# Funktionsdefinition:
ende()
{
        echo "Programm beendet"
        exit 0
}

# Funktion ende wird bei CTRL-C aufgerufen
trap 'ende' SIGINT

while true ; do
        echo "warte..."
        sleep 10
done
\end{verbatim}


\end{document}

